% Created 2022-02-13 Sun 22:20
% Intended LaTeX compiler: pdflatex
\documentclass[11pt]{article}
\usepackage[utf8]{inputenc}
\usepackage[T1]{fontenc}
\usepackage{graphicx}
\usepackage{grffile}
\usepackage{longtable}
\usepackage{wrapfig}
\usepackage{rotating}
\usepackage[normalem]{ulem}
\usepackage{amsmath}
\usepackage{textcomp}
\usepackage{amssymb}
\usepackage{capt-of}
\usepackage{hyperref}
\author{Arulselvan Madhavan}
\date{\today}
\title{}
\hypersetup{
 pdfauthor={Arulselvan Madhavan},
 pdftitle={},
 pdfkeywords={},
 pdfsubject={},
 pdfcreator={Emacs 27.2 (Org mode 9.4.4)}, 
 pdflang={English}}
\begin{document}

\tableofcontents

Jamila Lyiscott uses different variants of the English language to showcase how inequalities in the societies created isolated groups where people used language creatively without complying to the existing rules/norms/grammars and yet not compromising on the functional use of the language in anyway. Jamila Lyiscott switches between three different variants of the American English. The home version, school version and friends version. She calls attention to the racial disparity in the treatment of these versions. As Ahearn points out in the first chapter of her book, language, as performed by humans, goes beyond grammer. The communication is multimodal. In this video, Jamila Lyiscott, shows her use of the language varies depening on the social context she is in.

Jamila Lyiscott displays both linguistic competence and linguistic performance by switching between the version that follows the popular understanding of English grammer and to the versions that doesn't conform to the rigid grammar rules in the home and friends settings. Linguistic competence(Ahearn 2017, 17) is the abstract and usually unconscious knowledge that one has about the rules of a language, and linguitic performance is the act of putting the language into practice. When Jamila talks about why reflects at a lady's comment on her being articulate, she starts by showing her linguistic competence - using English language that follows the grammar that the lady had in her mind and most of the audience has in their mind. Then, she switches to linguistic performance when she wonders what being articulate means in her home and among friends. In the text, Ahearn provides Chomskyian definition of linguistic competence and linguistic performance and De Saussure's distinction between /langue/(language system in the abstract) and /parole/(every day speech) by using seater knitting as an analogy. Linguistic Competence would be knowing the abstract rule of knitting and linguistic performance would be how knitting is practiced.

Overall, Jamila Lyiscott, questions the racial disparity in the usage of her English language in the environment she grew up in. In this case, the abstract rules of the language/grammar became a tool through which racial disparity got introduced into the society. Jamila Lyiscott, by calling her trilingual, and treating all three variants of the English language as equal provides way for people who uses the less common versions of the language into the mainstream society
\end{document}
